\subsection{Motivation}
Let us consider a system made of $N$ bosonic modes that are described using the annihilators $a_1, \ldots, a_N$ which follow the commutation relations

\begin{eqnarray}
    \commutator{a_i}{a^\dagger_j} &=& \delta_{i j},\\
    \commutator{a_i}{a_j} &=& 0.
\end{eqnarray}

As we shall see afterwards, in order to study the time evolution of $a_i$, we need to evaluate, among other things, $\commutator{a_i}{H}$, where $H$ is the Hamiltonian of the system. In general, $H$ can be written as a polynomial of the operators $a_1, \ldots, a_N$ and of $a^\dagger_1, \ldots, a^\dagger_N$. We introduce the notation

\begin{equation}
    H = \poly{m}{a_1, \ldots, a_N; a^\dagger_1, \ldots, a^\dagger_N},
\end{equation}
meaning that the degree of $H$ is less or equal to $m$ (in the sense that, for each term of $H$, the \textbf{sum} of the powers of different operators is \textbf{less or equal} to $m$). Therefore, we need to know the properties of commutation relations between polynomials of bosonic annihilators and creators.


\subsection{Properties}

\subsubsection{Commutator Algebra}

In this part, we recall some usual properties of the commutator~\cite{commuator_wiki}. The proofs are given in~\autoref{proof-basic-commutator}.

\begin{theorem} \label{basic_commutator}
    The commutator $\commutator{\cdot}{\cdot}$ satisfies the following properties
    \begin{enumerate}
        \item $\commutator{\cdot}{\cdot}$ is anti-symmetric under exchange of the arguments.
        \item $\commutator{\cdot}{\cdot}$ is bilinear (i.e.\@ linear in each variable).
        \item $\forall A, B, C,\  \commutator{A}{B C} = \commutator{A}{B} C + B \commutator{A}{C}$ and $\commutator{B C}{A} = \commutator{B}{A} C + B \commutator{C}{A}$.
    \end{enumerate}
\end{theorem}

The last result has an interesting corollary if we consider a bosonic annihilator $a$.

\begin{corollary}
    Let $a$ be a bosonic annihilator, for each $n \in \mathbb{N}$, $\commutator{a}{a^{\dagger n}} = n a^{\dagger (n-1)}$ and $\commutator{a^n}{a^\dagger} = n a^{(n-1)}$
\end{corollary}

\subsubsection{Degree of Commutator of Polynomials}

The goal of this part is to state the result~\autoref{th_deg_comm}. This result will be very important in analysing the differential systems to solve and will allow us to distinguish between linear and non-linear systems. Even though this result is mentioned in the literature~\cite{quesada_deg}, the proof could not be found in the literature we read. The proof in~\autoref{proof-th_deg_comm} is, at the best of our knowledge, original.

\begin{theorem} \label{th_deg_comm}
    Let $p, m \in \mathbb{N}$, we have
    \begin{equation}
        \commutator{\poly{p}{a_1, \ldots, a_N; a^\dagger_1, \ldots, a^\dagger_N}}{\poly{m}{a_1, \ldots, a_N; a^\dagger_1, \ldots, a^\dagger_N}} = \poly{p+m-2}{a_1, \ldots, a_N; a^\dagger_1, \ldots, a^\dagger_N}
    \end{equation}

    where $\poly{q}{a_1, \ldots, a_N; a^\dagger_1, \ldots, a^\dagger_N}$ designates any polynomial of degree $q$.
\end{theorem}