Even though we had the opportunity to tackle many important and interesting questions during this project, many problems remain sadly unanswered. One such a question is the study of time-dependent inputs $a_{\text{in}}(t)$. The challenge arises from the difficulty of defining Green's functions for non-linear systems~\cite{non-lin-green, paper-off-non-lin-green}. This question is important if we desire to engineer more complex pulses than piece-wise constant inputs, even though piece-wise constant inputs are easier to optimize~\cite{emily}. It is possible that the mean methods can be used to solve the problem numerically. Furthermore, other theoretical questions remain open as highlighted in \autoref{tab:summ-table}. Even though, we have gained a greater understanding of the problem $(A B)_H \neq A_H B_H$ in \autoref{multi-heisenberg-multi}, we do not have yet a satisfying answer. A clear answer to this question will allow us to know when TEA is exact for an open system, for example. It will also tell us when correlations and moments are equal. 

Last but not least, these methods still need to be applied for experimental cases. 