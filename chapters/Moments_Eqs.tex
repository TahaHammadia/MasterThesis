We recall that the advantage of the Heisenberg picture is that it allows to follow a small number of interesting quantities, which does not have to scale in an exponential fashion with the number of modes. These quantities can be followed by writing differential equations on moments (i.e.\@ averages on monomials of annihilators and creators) or on cumulants (defined in~\autoref{cumulant-method}).

Let $N$ denote the number of  bosonic modes. It will turn out that we will find the differential equations (on moments or cumulants) by taking the average of the Heisenberg equations for the monomials of $a_i$ and $a^\dagger_i$, which can be written as $\poly{p}{a_1, \ldots, a_N; a^\dagger_1, \ldots, a^\dagger_N}$. Let us consider the Hamiltonian $H = \poly{m}{a_1, \ldots, a_N; a^\dagger_1, \ldots, a^\dagger_N}$. If we limit ourselves to unitary evolution, we find

\begin{eqnarray}
    \timeDeriv{\average{\poly{p}{a_1, \ldots, a_N; a^\dagger_1, \ldots, a^\dagger_N}}} &=& \ihbar \average{\commutator{\poly{p}{a_1, \ldots, a_N; a^\dagger_1, \ldots, a^\dagger_N}}{H}} \nonumber\\
    &=& \average{\poly{p + m - 2}{a_1, \ldots, a_N; a^\dagger_1, \ldots, a^\dagger_N}}.
\end{eqnarray}

Therefore, writing a closed differential system with monomials of degree less or equal to $p$ is possible if and only if $m \le 2$, i.e.\@ in the case of a quadratic Hamiltonian~\cite{elie-these, Dudas2023-xx}. States produced by quadratic Hamiltonians (i.e.\@ Gaussian evolution) from vacuum are called Gaussian states~\cite{elie-these, quesada-fast-sum-Gauss}. They are characterized by a Gaussian Wigner distribution~\cite{explo_quant, quesada-fast-sum-Gauss, walls_milburn}. Knowing exactly the moments for a non-quadratic Hamiltonian requires solving \textit{an infinite number of Heisenberg equations}. Such an evolution is called non-Gaussian. Therefore, closing the system of equation on moments requires using an approximation that allows to express higher order moments as a function of lower order moments.

In general, there are $(2N+1)^p$ monomials containing at most $p$ annihilators and creators. However, using bosonic commutation relations, we can limit ourselves to normal order (i.e.\@ we limit ourselves to polynomials having creators at the left and annihilators to the right). The moments of degree less or equal to $p$ can be put in the form $\average{\left(\prod_{i = 1}^N a^{\dagger j_i}_i\right) \left(\prod_{i = 1}^N a^{k_i}_i\right)}$ with $\sum_{i=1}^N j_i+k_i \le p$. We might be interested in knowing the number of independent Heisenberg equations for moments of order $p$. Let $1 \le q \le p$ denote the total degree of a monomial. To count the number of monomials of degree $q$, we can consider $a_i$ and $a^\dagger_i$ as boxes where we can put $q$ balls. The corresponding combinatorics problem corresponds to choosing $q$ balls amongst $q$ balls and $2N-1$ separators (that separate between the boxes), i.e.\@ $\binom{2N+q-1}{q}$ choices. Summing over the different $q$'s, we get in total $\sum_{q=1}^p \binom{2N+q-1}{q}$ monomials to follow. We can further reduce the number of moments to follow by imposing $\sum_{i=1}^N j_i \le \sum_{i=1}^N k_i$, roughly by half (in the sense that the number of moments is $\sim \frac{1}{2} \sum_{q=1}^p \binom{2N+q-1}{q} = O(p^{2N})$ for $p \rightarrow \infty$, $N$ constant and $\sim \frac{(2 N)^p}{2 p!}$ for $N \rightarrow \infty$, $p$ constant; see \autoref{misc-nbr-eq} for a proof of these statements). In conclusion, fixing the order of the operators in the moments reduces significantly the number of moments to follow.

For the remaining, we define: $\tau_H \equiv m - 2$, the ``Hamiltonian non-linearity''. This number measures the degree of non-linearity of the Hamiltonian.

\paragraph{Interesting Corollary:} Let us consider the value of~\autoref{exact_tea_expansion} for $O = a_i$. Since each commutation with $H$ adds $\tau_H$ to the degree of the polynomial, we have:

\begin{equation}
    \Comm{n}{H}{a_i} = \poly{1 + n \tau_H}{a_1, \ldots, a_N; a^\dagger_1, \ldots, a^\dagger_N}
\end{equation}

In particular, for $\tau_H \ge 1$ (or $H$ non-quadratic), the annihilator $a_i(t)$ cannot be expressed in terms of a polynomial of a finite degree of the initial ladder operators. This should be contrasted with the quadratic case where $a_i(t)$ is linear in the ladder operators (i.e.\@ $a_i(t) = \sum_{j=1}^N \left(F_{ij}(t) a_j(t) +G_{ij}(t) a^\dagger_j(t)\right)$ as in~\cite{elie-these}). Thus, there is a \textbf{fundamental} physical difference between quadratic and non-quadratic Hamiltonians. Quadratic Hamiltonians lead to new bosonic quasi-particles~\cite{ginzburg1958theory} while non-quadratic Hamiltonians lead to excitations of any arbitrary number of the initial particles (i.e.\@ $a_i(t)$ can contain for example $a_i$, $a^2_i$\ldots).
