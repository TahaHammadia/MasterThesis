The year 2025 marks the $100^{\text{th}}$ birthday of the formulation of the Heisenberg principle and the algebraic structure of quantum physics~\cite{history_quantum}. 2025 will be celebrated as the International Year of Quantum Science and Technology organized by the UNESCO~\cite{webpage_quantum_2025}. This quantum century has allowed humankind to develop technologies based on the understanding of the Universe provided by quantum physics ([classical] computers, telecommunications, medical imaging, nuclear energy, chemistry, metrology\ldots). Nowadays, a new application of quantum physics, which aims to leverage quantum information, is being developed: quantum computing.

Quantum computers aim to exploit the potential advantage provided by quantum information that is absent in classical information. Such an advantage can be formulated as the superposition principle~\cite{superposition_time}, the exponential size of the Hilbert space~\cite{schuld} or entanglement~\cite{aspect1982experimental}. Many objects have been explored for building quantum computers. Amongst them, we can cite: trapped ions~\cite{nature_atomic_qubits, compact_ion_trap, explo_quant}, cold atoms~\cite{Book-Lukacs} including Rydberg atoms~\cite{quantum_opti_rydberg}, diamond cavities~\cite{diamond}, supra-conducting circuits~\cite{google-supra, bbn-supra}\ldots This project focuses on light, which can be in the optical range or in the microwave range  (bosonic error correction in Alice \& Bob and quantum neuromorphic computing in Laboratoire Albert Fert).

The development of quantum computing devices based on light in all its forms, whether they are based on single photon sources~\cite{pascale_nature}, on linear optics~\cite{rawad_quandela}, on the generation of continuous states of light~\cite{review_xanadu} or on manipulating superposition of coherent states~\cite{elie_cat} has increased the hopes of seeing useful quantum computers in our lifetimes. Nevertheless, engineering such devices requires being able to simulate them beforehand. Since quantum computers are not mature enough to use them for simulation~\cite{feynman2018simulating}, it still needs to be done on classical computers. This becomes challenging as we approach applications where a quantum advantage is hoped for.

This project is part of the effort of simulating quantum devices based on bosons. This is important since many applications, like quantum machine learning and quantum neuromorphic computing~\cite{taha_quandela, markovic2020physics, Dudas2023-xx}, require to run the quantum device a considerable amount of times (for example once for each set of parameters to be tried). In particular, machine learning and neuromorphic tasks require using non-linearities to engineer, amongst others, powerful classification tools. Previously, this non-linearity was achieved \textit{via} measurements. However, current devices at Laboratoire Albert Fert can produce a Kerr effect. Even though analytical expressions can be found for self-Kerr and cross-Kerr (cf.\@ \autoref{app-analytic-exp}), adding non-diagonal terms makes the problem very challenging, thus the interest of this work. 

In this report, we study the evolution in the Heisenberg picture of bosonic modes evolving under some unitary evolution with some quantum jumps that simulate the interaction with a Markovian environment, or include the environment modes. We shall present some methods that aim at solving such systems. We start by presenting a overview of the problem and of the literature (cf.\@ \autoref{generalities-chapter}). Afterwards, we present our personal contribution (cf.\@ \autoref{main-chapter}). A great care will be taken to quantitatively compare the different methods (cf.\@ \autoref{sub-num-stab} and \autoref{comp-methds}). Additionally, we try to provide proofs for the different claims we make. In order not to hamper the reading of the report, the results and the proofs are demarcated clearly, allowing the hastened reader to skip them if needed. The report contains many appendices that clarify it and provide additional details. 