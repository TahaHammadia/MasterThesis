The development of quantum devices for quantum information and quantum neuromorphic computing leads to the necessity to simulate open bosonic systems. Such systems are characterized by the canonical bosonic commutation relations which have interesting properties that are derived. Solving the dynamics of open systems described by more than quadratic Hamiltonians, by unusual jump operators like two-photon loss or under an external drive is quite challenging. Attempts to simulate these systems \textit{via} the evolution of the density matrix, by imposing a cutoff in the Fock basis, are limited by its memory complexity which grows exponentially with the number of bosonic modes. This report undertakes part of the work by presenting the problem in a rigorous manner and leveraging the Heisenberg representation to propose and compare different methods of resolution that circumvent the memory complexity problem. These methods try to estimate the value of the moments, which describe the full dynamics. A great emphasis is put into writing correctly the Heisenberg representation for an open system. The complexities of the different methods will be evaluated. Furthermore, since the different methods are based on approximations, stability problems arise and need to be studied. We present a framework to study the stability of the numerical system and compare it for different cases between the different methods. Finally, analytical expressions are derived for some particular cases. Doing so allows to deepen the understanding of the resolution of differential equations on operators.