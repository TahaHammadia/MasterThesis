We have seen in \autoref{schro-pic} that solving the dynamics of the system in the Schrödinger picture requires a large memory. This can be very limiting since matrix operations need to be done on this representation, leading to a large time complexity. In this part, we change the representation of the dynamics in the hope that these problems will be solved.

In the Heisenberg representation, the dynamics are described by the evolution of the operators and a time-independent density matrix. This transformation is defined carefully in this section.

\subsubsection{Reminders Concerning the Heisenberg Picture}

In this part, the subscript or superscript $S$ refers to the Schrödinger picture, while $H$ refers to the Heisenberg picture. The evolution of a quantum state is given by the Kraus operators $K_k$~\cite{explo_quant, green_bible}
\begin{equation}
    \rho_S(t) = \sum_{k} K^{(S)}_k(t) \rho(0) K^{(S) \dagger}_k(t),
\end{equation}
with the normalization condition $\sum_k K^{(S) \dagger}_k(t) K^{(S)}_k(t)  = \mathbb{1}$ and the sum $\sum_{k}$ being potentially uncountable. This condition ensures that the trace of $\rho_S(t)$ is preserved. The Heisenberg representation is defined as~\cite{stack_exch_prod}

\begin{definition}[Heisenberg Representation]
    The Heisenberg representation of an operator $O_S(t)$ (with explicit time-dependence) is given by:
    \begin{equation}
        O_H(t) \equiv \sum_k K^{(S) \dagger}_k(t) O_S(t) K^{(S)}_k(t).
    \end{equation}
The Heisenberg representation of a density matrix is given by $\rho_H(t) \equiv \rho_S(0)$ (time-independent).
\end{definition}

The definition of the Heisenberg picture comes from the fact that \textbf{the averages of observables should not depend on the representation}. This can be generalized to any operator $O$ by taking linear combinations of $\frac{O+O^\dagger}{2}$ and $\frac{O-O^\dagger}{2i}$ which are observable. The average of the operator $O_S(t)$ (i.e. with explicit time-dependence) is given by
\begin{eqnarray}
    \average{O_S}(t) &=& \Tr{\rho_S(t) O_S(t)} = \sum_k \Tr{K^{(S)}_k(t) \rho_S(0) K^{(S) \dagger}_k(t) O_S(t)} \nonumber\\
    &=& \sum_k \Tr{\rho_S(0) K^{(S) \dagger}_k(t) O_S(t) K^{(S)}_k(t)} \nonumber\\
    &=& \Tr{\rho_S(0) \sum_k K^{(S) \dagger}_k(t) O_S(t) K^{(S)}_k(t)}.
\end{eqnarray}

We observe that $(A + \lambda B)_H = A_H + \lambda B_H$ and $\left(O^\dagger\right)_H = \left(O_H\right)^\dagger$. However, it is quite important to observe that in general $(A B)_H \neq A_H B_H$ unless the sum over $k$ is on one element (i.e.\@ closed system). In fact,

\begin{eqnarray}
    A_H(t) B_H(t) &=& \sum_k \sum_l K^{(S) \dagger}_k(t) A_S K^{(S)}_k(t) K^{(S) \dagger}_l(t) B_S K^{(S)}_l(t)\\
    (A B)_H(t) &=& \sum_k K^{(S) \dagger}_k(t) A_S B_S K^{(S)}_k(t).
\end{eqnarray}

Furthermore, the average $\average{A B}(t)$ is associated to $(A B)_H(t)$.

The interest of the Heisenberg picture is to study the evolution of average of operators, allowing us to follow a smaller number of quantities than required for solving the master equation~\autoref{master_eq}. For this purpose, we need to write the equation of evolution of the operators.

\begin{theorem}[Evolution equation in the Heisenberg Picture] \label{eq_evo_heisen}
    Under the Hamiltonian $H_S(t)$ and jump operators $L_n(t)$, the time evolution of $O_H(t)$, with $O_S$ time-independent, is given by
    \begin{equation} \label{equa-evo_heisen}
        \timeDeriv{O_H} = \left(\ihbar \commutator{O_S}{H_S(t)} + \frac{1}{2} \sum_n \left(L^\dagger_n(t) \commutator{O_S}{L_n(t)}  - \commutator{O_S}{L^\dagger_n(t)} L_n(t) \right)\right)_H.
    \end{equation}
\end{theorem}

\begin{proof}
    The master equation is given by:
    \begin{equation}
        \timeDeriv{\rho_S(t)} = \ihbar \commutator{H_S(t)}{\rho_S(t)} + \frac{1}{2} \sum_{n} \left(2 L_n(t) \rho_S(t) L_n^\dagger(t) - \anticomm{L_n^\dagger(t) L_n(t)}{\rho_S(t)}\right).
    \end{equation}
    Taking the average for the time-independent operator $O_S$, we find:
    \begin{eqnarray}
        \timeDeriv{\Tr{\rho_S(t) O_S}} &=& \ihbar \Tr{\commutator{H_S(t)}{\rho_S(t)}  O_S} + \frac{1}{2} \sum_{n} \left(2 \Tr{L_n(t) \rho(t) L_n^\dagger(t) O_S} - \Tr{\anticomm{L_n^\dagger(t) L_n(t)}{\rho(t)} O_S }\right) \nonumber\\
        &=& \ihbar \Tr{H_S(t) \rho(t) O_S} - \ihbar \Tr{\rho(t) H_S(t)  O_S}\nonumber\\
        &+& \frac{1}{2} \sum_{n} \left(2 \Tr{\rho(t) L_n^\dagger(t) O_S L_n(t)} - \Tr{L_n^\dagger(t) L_n(t) \rho(t) O_S} - \Tr{\rho(t) L_n^\dagger(t) L_n(t) O_S}\right)\nonumber\\
        &=& \ihbar \average{\commutator{O_S}{H_S(t)}} + \frac{1}{2} \sum_{n} \left(2 \average{L_n^\dagger(t) O_S L_n(t)} - \average{O_S L_n^\dagger(t) L_n(t)} - \average{L_n^\dagger(t) L_n(t) O_S}\right)\nonumber
    \end{eqnarray}
    We observe that: $2 L_n^\dagger O_S L_n - O_S L_n^\dagger L_n - L_n^\dagger L_n O_S = L_n^\dagger \left(O_S L_n - L_n O_S\right) + \left(L_n^\dagger O_S - O_S L_n^\dagger\right) L_n = L_n^\dagger \commutator{O_S}{L_n} - \commutator{O_S}{L_n^\dagger} L_n$. Furthermore, since our computation does not depend on the initial state $\rho(0)$, we deduce that:
    \begin{equation}
        \timeDeriv{O_H} = \left(\ihbar \commutator{O_S}{H_S(t)} + \frac{1}{2} \sum_n \left( L^\dagger_n \commutator{O_S}{L_n}  - \commutator{O_S}{L^\dagger_n} L_n \right)\right)_H
    \end{equation}
\end{proof}

\begin{remark}
    \autoref{eq_evo_heisen} implies that the evolution of an operator in Heisenberg picture can be computed using the standard bosonic commutation relations, since $O_S$ is a constant in \autoref{eq_evo_heisen}. This is important since commutation relations are not conserved for $t > 0$ (to see this consider $H = \hbar \omega a^\dagger a$ with the jump operator $L = \sqrt{\kappa} a$ where $\commutator{a(t)}{a^\dagger(t)} = e^{-\kappa t}$, see \autoref{misc-quad-one-loss} for a proof).
\end{remark}

\begin{remark}
    \autoref{equa-evo_heisen} does not need to have a closed form in $O_H$. It can depend on the Heisenberg representation of other operators. This is the crux of the problem of solving non-quadratic Hamiltonians (cf.\@~\autoref{intro-MF}).
\end{remark}


\paragraph{Example where $(A B)_H \neq A_H B_H$:} This example is from~\cite{stack_exch_prod}. Consider $H = \hbar \omega a^\dagger a$ with jump operators $L_1 = \sqrt{\kappa} a$ (spontaneous emission) and $L_2 = \sqrt{\kappa} e^{-\frac{\beta \hbar \omega}{2}} a^\dagger$ (thermal excitation). We can show that $a(t) = e^{-i \omega t -\frac{\gamma t}{2}} a(0)$ with $\gamma \equiv \kappa (1-e^{-\beta \hbar \omega})$. Thus: $a^\dagger(t) a(t) = e^{-\gamma t} a^\dagger(0) a(0)$. However, by writing and solving the differential equation on $n(t) \equiv \left(a^\dagger a\right)(t)$, we find: $n(t) = e^{-\gamma t} (n(0) - n_\infty) + n_\infty$, with $n_\infty$ corresponding to the Bose-Einstein distribution $n_\infty \equiv \frac{1}{e^{\beta \hbar \omega}- 1}$. Indeed, $n(t)$ represents the population which can go to a nonzero value while $a^\dagger(t) a(t)$ is a correlation that decays towards zero. \autoref{multi-heisenberg-multi} presents a sufficient condition that ensures that $(A B)_H = A_H B_H$. Even though the result is in itself not very practical, proving it gives a better understanding of the Heisenberg representation for an open system that can be seen by writing the equations of evolution for $(A B)_H$ and $A_H B_H$.

\paragraph{Comment on Bosonic Commutation Relations:} Even though $a(t)$ does not satisfy bosonic commutation relations for an open system and thus does not describe a boson in general, we show in this part that bosonic commutation relations are preserved. More specifically:

\begin{theorem}[Preservation of Bosonic Commutation Relations] \label{comm-preservation}
    We have for any open or closed bosonic system:
    \begin{eqnarray}
        \left(\commutator{a_i}{a_j}\right)(t) &=& 0\\
        \left(\commutator{a_i}{a^\dagger_j}\right)(t) &=& \delta_{i j}
    \end{eqnarray}
\end{theorem}

\begin{proof}
    We show more generally that the Heisenberg representation of a scalar $\lambda$ is constant. This proves the desired result since canonical bosonic commutators are scalar. We have
    \begin{equation}
        \lambda_H(t) = \sum_k K^{(S) \dagger}_k(t) \lambda K^{(S)}_k(t) = \lambda \sum_k K^{(S) \dagger}_k(t) K^{(S)}_k(t) = \lambda,
    \end{equation}
    where we used the normalization condition $\sum_k K^{(S) \dagger}_k(t) K^{(S)}_k(t)  = \mathbb{1}$.
\end{proof}

\begin{corollary}
    In the case of a \textbf{closed} system, we can write:
    \begin{eqnarray}
        \commutator{a_i(t)}{a_j(t)} &=& 0\\
        \commutator{a_i(t)}{a^\dagger_j(t)} &=& \delta_{i j}
    \end{eqnarray}
    This shows that $a_i(t)$ describes a boson for a closed system.
\end{corollary}

\begin{remark}
    We expect that adding the external input $a_{\text{in}}(t)$ and keeping the output $a_{\text{out}}(t)$, as done in a quantum Langevin equation, leads to a closed system where $(A B)_H = A_H B_H$. So far, this was not proven formally.
\end{remark}


\subsubsection{Case of Time-Independent Hamiltonian without Jump Operators}

In the case of a time-independent Hamiltonian without jump operators (i.e.\@ closed system), the Schrödinger and Heisenberg pictures of the aforementioned Hamiltonian are the same.

\begin{theorem}[Time-Independent Hamiltonian for a Closed System]
    For time-independent $H_S$: $H_S = H_H$.
\end{theorem}

\begin{proof}
    For $H_S$ time-independent: $U_S(t) = e^{-i\frac{H_S t}{\hbar}}$, which commutes with $H_S$.
\end{proof}

\begin{corollary}
    If $H_S$ is time-independent for a closed system, then $H_H$ is time-independent.
\end{corollary}

In the following, we will be interested in the operators that are not explicitly time-dependant, such as polynomials of ladder operators.

\begin{theorem}
    For $H_S$ and $O_S$ time-independent for a closed system, the time-evolution of $O_H$ is given by
    \begin{equation}
        O_H(t) = e^{i\frac{H_H t}{\hbar}} O_H(0) e^{-i\frac{H_H t}{\hbar}}
    \end{equation}
\end{theorem}

\begin{proof}
    For $H_S$ and $O_S$ time-independent, the equation of evolution of $O_H$ is given by
    \begin{equation} \label{equa-diff-closed}
        \timeDeriv{O_H(t)} = \ihbar \commutator{O_H(t)}{H_H}
    \end{equation}

    Since $H_H$ is time-independent, the solution of the differential equation is $O_H(t) = e^{i\frac{H_H t}{\hbar}} O_H(0) e^{-i\frac{H_H t}{\hbar}}$. This can be proven by showing that the right hand side \textit{satisfies the same differential equation} and \textit{has the same initial condition}.
\end{proof}

For the following, we will consider that all the operators are in the Heisenberg picture and drop the subscript $H$. In order to compute $O(t)$ explicitly, we introduce the following quantity.

\begin{definition}[Composition of Commutators]
    We define the sequence of operators $\Comm{n}{H}{O}$ by
    \begin{eqnarray}
        \Comm{0}{H}{O} &=& O\\
        \forall n \in \mathbb{N}, \Comm{n+1}{H}{O} &=& \commutator{H}{\Comm{n}{H}{O}}
    \end{eqnarray}
    which formally defines $\commutator{H}{\commutator{H}{\ldots\commutator{H}{O}}}$.
\end{definition}

Using this definition, we can write the time evolution of $O(t)$:

\begin{theorem} 
    For $H$ time-independent, we have
    \begin{equation} \label{exact_tea_expansion}
        O(t) = \sum_{n = 0}^\infty \frac{1}{n!} \left(\frac{i t}{\hbar}\right)^n \Comm{n}{H}{O(0)}
    \end{equation}
\end{theorem}

\begin{proof}
    We introduce the linear operator: $\Lambda : X \mapsto \ihbar \commutator{X}{H}$ (right hand side of~\autoref{equa-diff-closed}). Let us consider the linear differential equation: $\timeDeriv{X}(t) = \Lambda(X(t))$. The solution of this differential equation is $X(t) = \exp{t \Lambda}(X(0)) \equiv \sum_{n = 0}^\infty \frac{t^n}{n!} \Lambda^{(n)}(X(0))$, where $\Lambda^{(n)}$ designates the composition $n$ times. This is just a re-writing of~\autoref{exact_tea_expansion}.
\end{proof}

\begin{remark}
    In \autoref{exact_tea_expansion}, we can take $H$ both in the Heisenberg and the Schrödinger representations. Since we are computing its commutators with $O(0)$, $H$ should be considered in the Schrödinger picture. In particular, since all the operators are in the Schrödinger picture, \textbf{we can use the canonical commutation relations}.
\end{remark}